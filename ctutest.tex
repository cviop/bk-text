% arara: pdflatex: { synctex: yes }
% arara: makeindex: { style: ctuthesis }
% arara: bibtex

% The class takes all the key=value arguments that \ctusetup does,
% and a couple more: draft and oneside
\documentclass[twoside]{ctuthesis}

\ctusetup{
	preprint = \ctuverlog,
%	mainlanguage = english,
%	titlelanguage = czech,
	mainlanguage = czech,
	otherlanguages = {slovak,english},
	title-czech = {Přenos telemetrických dat z meteorologického balónu},
	title-english = {Telemetric Data Transmission from Meteorological Balloon},
	subtitle-czech = {},
	subtitle-english = {},
	doctype = B,
	faculty = F3,
	department-czech = {Katedra elektromagnetického pole},
	department-english = {Department of electromagnetic field},
	author = {Jakub Dvořák},
	supervisor = {Ing. Tomáš Kořínek, Ph.D.},
	supervisor-address = {Technická 2, \\ Praha 6},
	supervisor-specialist = {},
	fieldofstudy-english = {},
	subfieldofstudy-english = {},
	fieldofstudy-czech = {},
	subfieldofstudy-czech = {},
	keywords-czech = {slovo, klíč},
	keywords-english = {word, key},
	day = 20,
	month = 5,
	year = 2022,
	specification-file = {zav_prace.pdf},
%	front-specification = true,
%	front-list-of-figures = false,
%	front-list-of-tables = false,
%	monochrome = true,
%	layout-short = true,
}

\ctuprocess

\addto\ctucaptionsczech{%
	\def\supervisorname{Vedoucí}%
	\def\subfieldofstudyname{Studijní program}%
}

\ctutemplateset{maketitle twocolumn default}{
	\begin{twocolumnfrontmatterpage}
		\ctutemplate{twocolumn.thanks}
		\ctutemplate{twocolumn.declaration}
		\ctutemplate{twocolumn.abstract.in.titlelanguage}
		\ctutemplate{twocolumn.abstract.in.secondlanguage}
		\ctutemplate{twocolumn.tableofcontents}
		\ctutemplate{twocolumn.listoffigures}
	\end{twocolumnfrontmatterpage}
}

% Theorem declarations, this is the reasonable default, anybody can do what they wish.
% If you prefer theorems in italics rather than slanted, use \theoremstyle{plainit}
\theoremstyle{plain}
\newtheorem{theorem}{Theorem}[chapter]
\newtheorem{corollary}[theorem]{Corollary}
\newtheorem{lemma}[theorem]{Lemma}
\newtheorem{proposition}[theorem]{Proposition}

\theoremstyle{definition}
\newtheorem{definition}[theorem]{Definition}
\newtheorem{example}[theorem]{Example}
\newtheorem{conjecture}[theorem]{Conjecture}

\theoremstyle{note}
\newtheorem*{remark*}{Remark}
\newtheorem{remark}[theorem]{Remark}

\setlength{\parskip}{5ex plus 0.2ex minus 0.2ex}

% Abstract in Czech
\begin{abstract-czech}
Aaaabstrakt

\end{abstract-czech}

% Abstract in English
\begin{abstract-english}
Abstract
	
\end{abstract-english}

% Acknowledgements / Poděkování
\begin{thanks}
Děkuji vedoucímu Tomáši Kořínkovi za cenné rady a pomoc při realizaci práce. Děkuji Ing. Martinu Motlovi za pomoc s vypouštěním sondy. (tmobile tracker)

\end{thanks}

% Declaration / Prohlášení
\begin{declaration}
	Prohlašuji, že jsem tuto práci vypracoval samostatně s~použitím literárních pramenů a~informací, které cituji a~uvádím v~seznamu použité literatury a~zdrojů informací.

V Praze, \ctufield{day}.~\monthinlanguage{title}~\ctufield{year}
\end{declaration}

% Only for testing purposes
\listfiles
\usepackage[pagewise]{lineno}
\usepackage{lipsum,blindtext}
\usepackage{mathrsfs} % provides \mathscr used in the ridiculous examples

\begin{document}

\maketitle

\chapter{Úvod}

\section{Cíl práce}

Tato práce ze zabývá vývojem a realizací pyčo.

\section{Způsob řešení}

A dělal jsem to takhlehehe.


a
\chapter{Návrh experimentu}
a

\section{Šíření vln ve troposféře}
a

\section{Měřená data}
a

\section{Součásti experimentu}
a

\chapter{Návrh systému}
a
\section{Požadavky}
a
\section{Hardware}
a
\section{Software}
a



\chapter{Realizace}
a
\section{Hardware}
a
\section{Firmware}
a
\section{Software}
a
\section{Mechanická zástavba}
a
\section{Testování}
a




\chapter{Experiment}
a
\section{Průběh experimentu}
a
\section{Naměřená data}
a


\chapter{Výsledky}
a

\section{Zpracování dat}
a

\section{Výstup z experimentu}
a

\section{Vizualizace dat}
a




\chapter{Závěr}
a
\section{Shrnutí experimentu}
a
\section{Možná vylepšení}
a

\begin{table}
\begin{ctucolortab}
\begin{tabular}{cc}
\bfseries Foo & \bfseries Bar \\\Midrule
foo1 & bar1 \\
foo2 & bar2
\end{tabular}
\end{ctucolortab}
\caption{Foobar.}
\label{tab:foobar}
\end{table}

\begin{figure}
\includegraphics[width=0.4\textwidth]{ctu_logo_black}
\caption{Black logo of the CTU in Pragueueue.}
\end{figure}

\begin{figure}[!t]
\includegraphics[width=0.4\textwidth]{ctu_logo_blue}
\caption{Blue logo of the CTU in Pragueueue.}
\end{figure}

\chapter{Conclusions}

\section{Test --- this is just a little test of something in the table of contents}

\subsection{Yes, table of contents}

\begin{theorem}\begin{enumerate} \item Bla \item Blo \end{enumerate} \end{theorem}


Lorem ipsum dolor sit amet, consectetur adipiscing elit. Duis interdum facilisis urna, at tincidunt leo consectetur non. Maecenas bibendum mi vitae libero pharetra, ac ullamcorper nulla pellentesque. Sed sit amet massa nunc. Aenean placerat a est sodales sagittis. Quisque purus nibh, auctor ut consectetur at, suscipit non erat. Donec condimentum porttitor risus, vitae fringilla lectus tincidunt nec. Nulla leo quam, commodo eu ornare non, iaculis sed nulla. Duis gravida lacus quis purus sodales, vitae malesuada justo ultricies. Vestibulum nisl nulla, commodo non pellentesque a, fringilla a risus. Ut quis magna nulla. Mauris vitae ultricies ante, in consectetur justo. 

\medskip

\begin{proof}\begin{enumerate} \item[8] Bla \item Blo \end{enumerate} \end{proof}

\appendix

\printindex

\appendix

\bibliographystyle{amsalpha}
\bibliography{ctutest}

\ctutemplate{specification.as.chapter}

\end{document}