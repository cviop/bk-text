% arara: pdflatex: { synctex: yes }
% arara: makeindex: { style: ctuthesis }
% arara: bibtex

% The class takes all the key=value arguments that \ctusetup does,
% and a couple more: draft and oneside
\documentclass[twoside]{ctuthesis}

\ctusetup{
	preprint = \ctuverlog,
%	mainlanguage = english,
%	titlelanguage = czech,
	mainlanguage = czech,
	otherlanguages = {slovak,english},
	title-czech = {Přenos telemetrických dat z meteorologického balónu},
	title-english = {Telemetric Data Transmission from Meteorological Balloon},
	subtitle-czech = {},
	subtitle-english = {},
	doctype = B,
	faculty = F3,
	department-czech = {Katedra elektromagnetického pole},
	department-english = {Department of electromagnetic field},
	author = {Jakub Dvořák},
	supervisor = {Ing. Tomáš Kořínek, Ph.D.},
	supervisor-address = {Technická 2, \\ Praha 6},
	supervisor-specialist = {},
	fieldofstudy-english = {},
	subfieldofstudy-english = {},
	fieldofstudy-czech = {},
	subfieldofstudy-czech = {},
	keywords-czech = {slovo, klíč},
	keywords-english = {word, key},
	day = 20,
	month = 5,
	year = 2022,
	specification-file = {zav_prace.pdf},
%	front-specification = true,
%	front-list-of-figures = false,
%	front-list-of-tables = false,
%	monochrome = true,
%	layout-short = true,
}

\ctuprocess

\addto\ctucaptionsczech{%
	\def\supervisorname{Vedoucí}%
	\def\subfieldofstudyname{Studijní program}%
}

\ctutemplateset{maketitle twocolumn default}{
	\begin{twocolumnfrontmatterpage}
		\ctutemplate{twocolumn.thanks}
		\ctutemplate{twocolumn.declaration}
		\ctutemplate{twocolumn.abstract.in.titlelanguage}
		\ctutemplate{twocolumn.abstract.in.secondlanguage}
		\ctutemplate{twocolumn.tableofcontents}
		\ctutemplate{twocolumn.listoffigures}
	\end{twocolumnfrontmatterpage}
}

% Theorem declarations, this is the reasonable default, anybody can do what they wish.
% If you prefer theorems in italics rather than slanted, use \theoremstyle{plainit}
\theoremstyle{plain}
\newtheorem{theorem}{Theorem}[chapter]
\newtheorem{corollary}[theorem]{Corollary}
\newtheorem{lemma}[theorem]{Lemma}
\newtheorem{proposition}[theorem]{Proposition}

\theoremstyle{definition}
\newtheorem{definition}[theorem]{Definition}
\newtheorem{example}[theorem]{Example}
\newtheorem{conjecture}[theorem]{Conjecture}

\theoremstyle{note}
\newtheorem*{remark*}{Remark}
\newtheorem{remark}[theorem]{Remark}

\setlength{\parskip}{5ex plus 0.2ex minus 0.2ex}

% Abstract in Czech
\begin{abstract-czech}
Aaaabstrakt

\end{abstract-czech}

% Abstract in English
\begin{abstract-english}
Abstract
	
\end{abstract-english}

% Acknowledgements / Poděkování
\begin{thanks}
Děkuji vedoucímu Tomáši Kořínkovi za cenné rady a pomoc při realizaci práce. Děkuji Ing. Martinu Motlovi za pomoc s vypouštěním sondy. (tmobile tracker)

\end{thanks}

% Declaration / Prohlášení
\begin{declaration}
	Prohlašuji, že jsem tuto práci vypracoval samostatně s~použitím literárních pramenů a~informací, které cituji a~uvádím v~seznamu použité literatury a~zdrojů informací.

V Praze, \ctufield{day}.~\monthinlanguage{title}~\ctufield{year}
\end{declaration}

% Only for testing purposes
\listfiles
\usepackage[pagewise]{lineno}
\usepackage{lipsum,blindtext}
\usepackage{mathrsfs} % provides \mathscr used in the ridiculous examples

\begin{document}

\maketitle

\chapter{Úvod}


\chapter{Cíl práce}
Tato práce ze zabývá...\\
výroba sondy schopná měřit podmínky ve tropo a posílat je na zem, měření příchozího signálu na zemi, vyrobit model šíření

	\section{Šíření vln ve troposféře}
	jak to funguje, na čem to závisí (přešíst literaturu)

	\section{Způsob řešení / návrh experimentu}
	naměření dat z tropo a naměření dat na zemi a kombinace do modelu šíření vlny
		\subsection{Měřená data}
		jaká data budou měřena 

	\section{Součásti experimentu}
	co je potřeba udělat - hw, firmware, sw, mechaniku, naměření dat, naměření charakteristik antény, zpracování dat


\chapter{Návrh systému}
	\section{Požadavky}
	520 g, telemetrie, teplota, tlak vlhkost, gps, fungování do -40 - baterky, kompaktnost

	\section{Hardware}
	senzory do -40, nízký tlak, dosah 30+ km
		\subsection{Samostatná deska plošných spojů}
		kompaktní, spolehlivé, obtížné na debug, časové náročné
		\subsection{Vývojové moduly}
		snadné na vývoj a odladění, snadná změna zapojení pří psaní kódu

	\section{Software}
	čtení ze senzorů, parsování dat, posílání na zem a na sd kartu, odolnost, měření náklonu - jak?
		\subsection{Měření náklonu sondy}
		Acc, vektor mag pole, kalmanův filtr, co bylo použito



\chapter{Realizace}
	\section{Elektronika}
		\subsection{Testování modulů}
		měření odběru, energie pro poslání dat

		\subsection{Realizace elektroniky}
		Navrženy shieldy, nakresleny modely. Spínaný zdroj + LDO, ochrany vstupů, volné GPIO na rozšíření
		%...bylo využito vývojových modulů - desek plošných spojů s již osazenými senzory a měřícími čipy. Tyto desky byly osazeny také minimem potřebných součástek, zajišťujících správné fungování čipů, například blokovacích kondenzátorů umístěných v bezprostřední blízkosti čipů. 

		%Elektronika sondy byla stavěna kolem vývojového kitu Nucleo od firmy \textit{ST Microelectronics}. Jedná se o PCB s mikroprocesorem a minimem součástek, nutných pro správné fungování procesoru. Součástí desky je také zdroj pro napájení čipu a programátor. Jednotlivé piny mikroprocesoru jsou vyvedeny na pinové lišty na kraji desky a slouží k propojení s moduly, používanými v sondě. Výhodou tohoto řešení ve fázi vývoje je snadná záměna zapojení modulů a rychlé odstranění chyb způsobené při výběru komunikačních pinů. Po odladění elektronického propojení mikroprocesoru s moduly senzorů

	\section{Mechanická zástavba}
	model PCB, model sondy, iterace, odlehčování

	\section{Firmware}
	výstřizky z driverů, sample GPS dat, parsovací funkce, změřené minimum accelerace v z-ose, sešití dat, watchdog, reset při erroru

	\section{Software}
	parsování příchozích dat, doplnění NMEA zprávy pro tracker, python - parsování a přepočítání souřadnic, zobrazení na mapě, zobrazení v terminálu

	\section{Testování a měření}
	směrová charakteristika, teplotní odolnost, proudový odběr telitu 




\chapter{Experiment}
příjem dat, umístění antény, nastavení spektráku
	\section{Průběh experimentu}
	jak to probíhalo, co se stalo, proč sonda přestala vysílat, proč doletěla jen do 17 km, nalezení pomocí sondy čhmú, sundání sondy

	\section{Naměřená data}
	co bylo na SD kartě, výsledky měření - čístě změřená data



\chapter{Výsledky}
	\section{Zpracování dat}
	zkombinovat data ze země a data ze strato, vzorečky, určit refrakci, výkonnovou bilanci podle podmínek, vzít v potaz směrovou charstiku. vyrobit model šíření, grafy

	\section{Výstup z experimentu}
	výsledky, co bylo změřeno a zjištěno



\chapter{Závěr}
	\section{Shrnutí experimentu}
	co se povedlo, co se nepovedlo. Vyrobil jsem sondu a sw, přestala vysílat - proč? 

	\section{Možná vylepšení}
	malé pcb bez modulů, optimalizace sw, nepoužívat HAL, programovat přes registry, měření náklonu sondy, častější posílání dat, nezávislost na GPS




\appendix

\printindex

\appendix

\bibliographystyle{amsalpha}
\bibliography{ctutest}

\ctutemplate{specification.as.chapter}

\end{document}